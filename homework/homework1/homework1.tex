\documentclass{article}
\usepackage{enumerate}
\usepackage{amsmath}
\usepackage{amssymb}
\usepackage{geometry}
\usepackage{ulem}
\usepackage{indentfirst}
\geometry{left=3.0cm,right=3.0cm,top=3.0cm,bottom=4.0cm}
\renewcommand{\thesection}{\arabic{section}.}
\title{VW110 Hausaufgabe 1}
\author{Liu Yihao 515370910207}
\date{}

\begin{document}
\maketitle

\section{Schreiben Sie die entsprechenden Sprachen auf.}

\begin{minipage}{0.48\linewidth}
\begin{tabular}{lcl}
China & $\longrightarrow$ & Chinesisch \\
Deutschland & $\longrightarrow$ & Deutsch \\
Ukraine & $\longrightarrow$ & Ukrainisch \\
England & $\longrightarrow$ & Englisch \\
Russland & $\longrightarrow$ & Russisch \\
\end{tabular}

\end{minipage}
\hfill
\begin{minipage}{0.48\linewidth}
\begin{tabular}{lcl}
Frankreich & $\longrightarrow$ & Französisch \\
Japan & $\longrightarrow$ & Japanisch \\
USA & $\longrightarrow$ & Englisch \\
Türkei & $\longrightarrow$ & Türkisch \\
Portugal & $\longrightarrow$ & Portugiesisch \\
\end{tabular}
\end{minipage}

\section{Ergänzen Sie bitte den Dialog.}

$\ast$ Guten Tag! Mein \uline{Name} ist Anna Schneider. \uline{Wie} heißen Sie?

$\diamond$ Guten \uline{Tag}, \uline{Frau} Schneider. Ich \uline{heiße} Peter Schulz.

$\ast$ \uline{Woher} kommen Sie?

$\diamond$ Ich komme aus \uline{der} Schweiz. Und \uline{Sie}?

$\ast$ Ich komme \uline{aus} Deutschland, \uline{aus} Hamburg.

\section{Schreiben Sie zuerst Ihren Steckbrief. Dann schreiben Sie ganze Sätze.}

\begin{tabular}{ll}
Familienname: & Liu \\
Vorname: & Yihao \\
Land: & China \\
Stadt: & Shanghai \\
Sprachen: & Chinesisch, Englisch und ein bisschen Deutsch \\
\end{tabular}
\\

Mein Familienname ist Liu.

Mein Vorname ist Yihao.

Ich komme aus China, aus Shanghai.

Ich spreche Chinesisch, Englisch und ein bisschen Deutsch.

\end{document}
